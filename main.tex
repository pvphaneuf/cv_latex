\documentclass[10pt]{article}

%% Define some new colors
\usepackage{xcolor}
\definecolor{mBlue}{RGB}{51, 77, 167}
\newcommand{\blue}[1]{{\color{mBlue}#1}}

%% Page and text formatting
\usepackage[left=1.0in, right=1.0in, top=1.0in, bottom=1.0in]{geometry} % margins
\usepackage{setspace}
\singlespacing % No more than 6 lines of text per inch
\usepackage{amsmath, amsfonts}
\usepackage[T1]{fontenc}
\usepackage[utf8]{inputenc}
\usepackage{helvet}
\usepackage{todonotes}
\usepackage{lastpage}
% \usepackage{lmodern}

%% Make the sections start with (a), (b), etc.
\renewcommand\thesection{(\alph{section})}
\renewcommand\thesubsection{(\alph{subsection})}

%% Set up footer
\usepackage{fancyhdr}
\fancypagestyle{CVfooter}
{
 \lhead{}
 \chead{}
 \rhead{}
 \lfoot{\sffamily\small{Patrick V.\ Phaneuf}}
 \cfoot{\sffamily\small{\today}}
 \rfoot{\sffamily\small{\thepage/\pageref*{LastPage}}}
 \renewcommand{\headrulewidth}{0.0pt}
 \renewcommand{\footrulewidth}{0.5pt}
}

%% Set up hyperlinks
\usepackage[linktoc=page]{hyperref}
\hypersetup{
  colorlinks = true,
  urlcolor = mBlue,
  citecolor = mBlue,
  linkcolor = mBlue,
}

%% Set up citations and bibliography
\usepackage{bibunits}
\usepackage[sort&compress,super]{natbib}
\defaultbibliographystyle{apsrev4-2}
\setcitestyle{comma}
\setlength{\bibsep}{0pt}
\renewcommand{\bibnumfmt}[1]{\ \ \ #1.}
\renewcommand\refname{\vspace{-7mm}}

%% This allows us to start a bibliography with arbitrary number
\usepackage{etoolbox}
\makeatletter
\newcommand*{\newbibstartnumber}[1]{%
  \apptocmd{\thebibliography}{%
    \global\c@NAT@ctr #1\relax
    \addtocounter{NAT@ctr}{-1}%
  }{}{}%
}
\makeatother

%% Reduce whitespace around lists (globally)
\usepackage{enumitem}
\setlist[itemize]{nosep,left=0pt}
\setlist[enumerate]{nosep}
\setlist[description]{nosep}

%% Formatting for sections
\usepackage[compact]{titlesec}
\titleformat*{\section}{\normalsize\bfseries}
\titlespacing*{\section}{0mm}{2mm}{1mm}[0mm] % left top bottom right
\titleformat*{\subsection}{\normalsize\bfseries}
\titlespacing*{\subsection}{0mm}{2mm}{1mm}[0mm]
\titleformat*{\subsubsection}{\normalsize\bfseries\itshape}
\titlespacing*{\subsubsection}{0mm}{1mm}{0mm}[0mm]

\begin{document}
\pagestyle{CVfooter}

{\fontfamily{phv}\selectfont{
\noindent\begin{tabular*}{\textwidth}{@{\extracolsep{\fill}}l r}
\noindent\textbf{Patrick V.\ Phaneuf} & \textbf{Curriculum Vitae} \\
Ph.D. & Bioinformatics and Systems Biology  \\
University of California, San Diego & Email: \href{mailto:pphaneuf@eng.ucsd.edu}{pphaneuf@eng.ucsd.edu} \\
9500 Gilman Drive \\
La Jolla, CA 92093, USA  \\
\hline
\end{tabular*}

\section*{EDUCATION AND TRAINING}
Ph.D. Bioinformatics and Systems Biology, 2016--2021, University of California, San Diego, CA \\
M.Sc. Computer Science, 2014--2016, University of California, San Diego, CA \\
Software and Firmware Engineer, 2010--2014, Thermo Fisher Scientific, Eugene, OR \\
B.Sc. Computer Engineering, \textit{cum laude}, 2004--2009, Mercer University, Macon, GA

\section*{RESEARCH, TECHNICAL, AND PROFESSIONAL EXPERIENCE}
\textbf{Systems Biology Research Group (SBRG)}, Postdoc advised by Prof. Bernhard Palsson, 2021 -- present, Ph.D. and M.Sc. candidate advised by Dr. Adam Feist, 2014 -- 2021
\begin{itemize}
\item\textbf{Data-driven strain design using aggregated ALE data}, 2019 -- present \\
Designing application-focused sequence changes to the \textit{E. coli} K-12 genome using a novel multi-dimensional adaptive laboratory evolution dataset, a multi-scale graph-based genome annotation engine, and structural biology tools. \href{https://doi.org/10.1021/acssynbio.1c00337}{10.1021/acssynbio.1c00337}
\item\textbf{Graph-based genome annotation engine and mutation enrichment}, 2018 -- present \\
Applying enrichment methods to mutations described by a novel multi-scale annotation framework and a consolidated set of experimental evolution conditions to automate key mutation identification and to deconvolute mutation selection pressures. \href{https://doi.org/10.1186/s12864-020-06920-4}{10.1186/s12864-020-06920-4}
\item\textbf{ALEdb: a web-accessible experimental evolution mutation and conditions database}, 2014 -- present \\
Developed first version of a web-accessible public database consolidating and reporting on mutations from experiments executed by researchers across the field of experimental evolution (\href{https://aledb.org/}{aledb.org}). Managing further development, maintenance, and growth by a team of software engineers and data curator (1--3). Major milestones include containerization, cloud-deployment, and the integration of 6000 samples (75000 unique mutations), . \href{https://doi.org/10.1093/nar/gky983}{10.1093/nar/gky983}
\item\textbf{Automated mutation calling pipeline for large-scale ALE experiments}, 2014 -- present \\
Developed initial versions of an automated mutation calling pipeline used by the Novo Nordisk Foundation Center for Biosustainability (CfB). Managing further development and maintenance by a team of software engineers (1--3). Major milestones include automated quality control, metadata workflow, large-scale sample set auto-organization, ensemble mutation calling, containerization, and cloud-enabled parallelization.
\item\textbf{ALE group software team lead}, 2016 -- 2021\\
Managed a team of full-time software engineers (1--3) and data curator that developed and maintained 4 different products. Interfaced with multiple research, software, and informatics groups within the CfB. Lead efforts in adoption of cloud services for compute and storage within the SBRG.
\end{itemize}
\vspace{1mm}
\textbf{Software and Firmware Engineer, Thermo Fisher Scientific}, 2010 -- 2014 \\
Developed system control software and firmware for the Attune® Acoustic Focusing Cytometer and Attune® Autosampler projects. Experience the complete product life-cycle for both a software and hardware product through a combination of the first and second iteration of the Attune® and its Autosamplers.

\section*{SELECTED PUBLICATIONS}
\vspace{2mm}
\begin{bibunit}
\nocite{apsrev42Control}
\nocite{Phaneuf2021-qd,Phaneuf2020-ql-,Phaneuf2019-to,Rychel2020-vi,Lamoureux2020-so,Sandberg2020-gr,Heckmann2020-ia,Du2020-od,Anand2019-yy,Lloyd2019-sn,Anand2019-et,Guzman2018-wp,Pekar2018-dr,Yang2019-eb,Mundhada2017-rv}
\putbib[publications]
\end{bibunit}
}
\end{document}